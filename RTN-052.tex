\documentclass[OPS,authoryear,toc,lsstdraft]{lsstdoc}
\input{meta}

% Package imports go here.

% Local commands go here.

%If you want glossaries
%\input{aglossary.tex}
%\makeglossaries

\title{Charge to the Data Release Board}

% Optional subtitle
% \setDocSubtitle{A subtitle}

\author{%
Colin Slater, Leanne Guy
}

\setDocRef{RTN-052}
\setDocUpstreamLocation{\url{https://github.com/lsst/rtn-052}}

\date{\vcsDate}

% Optional: name of the document's curator
% \setDocCurator{The Curator of this Document}

\setDocAbstract{%
This document provides the charge to the Rubin Observatory Legacy Survey of Space
and Time (LSST) Data Release Board (DRB), a Rubin Operations internal group responsible for the
assessment of Rubin Observatory's schedule to make annual data releases.
}

% Change history defined here.
% Order: oldest first.
% Fields: VERSION, DATE, DESCRIPTION, OWNER NAME.
% See LPM-51 for version number policy.
\setDocChangeRecord{%
  \addtohist{1}{2024-03-01}{Initial Draft}{Colin Slater}
}


\begin{document}

% Create the title page.
\maketitle
% Frequently for a technote we do not want a title page  uncomment this to remove the title page and changelog.
% use \mkshorttitle to remove the extra pages

% ADD CONTENT HERE
% You can also use the \input command to include several content files.


\section{Rationale}

Production of the data release is a process that spans many months and requires a large investment
of time, staff effort, and compute resources. These factors necessitate a policy-level system to
ensure that the results of a DRP processing will maximize the scientific return from the data by
determining an appropriate scope of inputs, software components and versions, and output products
that will meet the needs of the science users.

This optimization is made challenging by the time required to produce a data release, which can make
it impractical to redo significant amounts of the processing if problems are discovered late in the
process. Therefore it is important that most decision-making take place at the start of the release
process, since the ability to make changes later during the release process is significantly
constrained. In the highly-constrained environment near the target release date, decisions regarding
the release contents and the release schedule may have significant impact on both the science users
and the observatory operational teams. The purpose of the DRB is to provide a mechanism for weighing
these wide-ranging tradeoffs and deciding how the release should proceed.

\section{Charge}

The DRB is responsible for setting the scope, schedule, and resources available for the data release
production. The DRB is expected to delegate daily management responsibility to a campaign committee,
but retains the overall authority and responsibility for the release.

\section{Composition}

The DRB is chaired by the Head of Science, and includes associate directors of all departments.

\section{Activities}

The DRB meets at the start of the DRP process to decide on the input dataset, the initial pipelines
version for precursor run testing, and the proposed schedule for processing and release. After an
initial precursor processing is performed on a small subset of the data, a period of validation
begins, and ends with proposals from the Verification and Validation team and Science Pipelines to
either begin the full processing or make further changes to the plan. The DRB is responsible for
approving the start of the full processing.

Once the production has begun, the process can continue according to the plan without further
intervention by the DRB. Day-to-day operational management is conducted by a campaign committee,
with representation from the technical teams involved in the production. This campaign committee
meets frequently to carry out the campaign as planned by the DRB.

However, if issues arise during production that potentially impact the scope, schedule, or resource
usage of the campaign, the campaign committee alerts the DRB, and the DRB decides on a best course
of action. For example, this could be due to particular algorithms or table columns that are found
to contain degraded results, unexpected delays in processing, or additional steps that are necessary
to enable scientific use. Minor versions of these problems are expected and can often be resolved by
the production teams and the campaign committee, but the DRB's input is critical when problems are
significant enough that a schedule delay or additional compute resources would be required, or
alternatively, the release of data with known caveats.

These mitigation activities can be time critical, and the DRB must be available to choose between
possible mitigations on short timescales. Delays in decision making can rapidly use up schedule
margin and cause a day-for-day delay in the release.



\appendix
% Include all the relevant bib files.
% https://lsst-texmf.lsst.io/lsstdoc.html#bibliographies
\section{References} \label{sec:bib}
\renewcommand{\refname}{} % Suppress default Bibliography section
\bibliography{local,lsst,lsst-dm,refs_ads,refs,books}

% Make sure lsst-texmf/bin/generateAcronyms.py is in your path
\section{Acronyms} \label{sec:acronyms}
\input{acronyms.tex}
% If you want glossary uncomment below -- comment out the two lines above
%\printglossaries




\end{document}
